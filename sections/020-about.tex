%!TEX root = ../beamer-ambitious-example.tex
\section{テーマについて}
\subsection{ブロック}
\begin{frame}{\insertsubsection}
  なんかすごい手法\cite{Ono:2010:LVAICA,Ono:2011:WASPAA,Ono:2012:IWAENC}
  \begin{columns}[T,onlytextwidth]
    \column{.5\linewidth}
      \begin{block}{block}
        block の色は\structure{\ruby[g]{群青}{ぐんじょう}色}
      \end{block}
      \begin{exampleblock}{exampleblock}
        example block の色は\textcolor{example text.fg}{なんかよくわからん緑色}
      \end{exampleblock}
      \begin{alertblock}{alertblock}
        alert block の色は\textcolor{alerted text.fg}{緋色}
      \end{alertblock}
    \column{.5\linewidth}
      \ambiset{block=fill}
      \begin{block}{block}
        block の色は\structure{\ruby[g]{群青}{ぐんじょう}色}
      \end{block}
      \begin{exampleblock}{exampleblock}
        example block の色は\textcolor{example text.fg}{なんかよくわからん緑色}
      \end{exampleblock}
      \begin{alertblock}{alertblock}
        alert block の色は\textcolor{alerted text.fg}{緋色}
      \end{alertblock}
      ブロックの背景を塗りつぶすこともできます
  \end{columns}
\end{frame}

{
  \setbeamertemplate{frametitle}[hidenumber]
  \begin{frame}[noframenumbering]{\insertsubsection}
    ページ番号を一時的に消すこともできます
    \uncover<2->{ページ数も矛盾しません!}
  \end{frame}
}

\subsection{Itemize}
\begin{frame}{\insertsubsection}
    見てくださいこのかわいいラベルたち
    \begin{itemize}
      \item[\pros] Pros
      \item[\cons] Cons
      \item[\todo] todo的なやつ
      \item[\done] todo完了的なやつ
      \item[\new] 新しいやつ
      \item[\goal] 「そこで,」的なやつ
    \end{itemize}
\end{frame}

\newcommand{\Obs}{\bm{x}}
\newcommand{\Src}{\bm{s}}
\newcommand{\Mix}{\bm{A}}
\newcommand{\freq}{f}
\newcommand{\tframe}{t}
\newcommand{\ft}{\freq\tframe}
\begin{frame}{数式の装飾}
  \begin{equation*}
    \tcboxmath[
        enhanced,
        frame hidden,
        interior hidden,
        size=minimal,
        underlay={\onslide<2->{\fill[structure.fg!20] (frame.south west) rectangle (frame.north east);}},
        overlay={\onslide<2->{\draw[structure.fg,<-] ([yshift=-1mm]frame.south) to[out=-90,in=0] +(-3mm,-4mm) node[left] {観測信号};}

        }
    ]{\Obs _{\ft}}
    =
    \tcboxmath[
        enhanced,
        frame hidden,
        interior hidden,
        size=minimal,
        underlay={\onslide<4->{\fill[structure.fg!20] (frame.south west) rectangle (frame.north east);}},
        overlay={\onslide<4->{\draw[structure.fg,<-] ([yshift=-1mm]frame.south) to[out=-90,in=180] +(3mm,-8mm) node[right] {混合行列};}

        }
    ]{\Mix _{\freq}}
    \tcboxmath[
        enhanced,
        frame hidden,
        interior hidden,
        size=minimal,
        underlay={\onslide<3->{\fill[structure.fg!20] (frame.south west) rectangle (frame.north east);}},
        overlay={\onslide<3->{\draw[structure.fg,<-] ([yshift=-1mm]frame.south) to[out=-90,in=180] +(3mm,-4mm) node[right] {音源信号};}

        }
    ]{\Src _{\ft}}
  \end{equation*}
\end{frame}
