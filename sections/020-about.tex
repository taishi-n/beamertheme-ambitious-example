%!TEX root = ../beamer-ambitious-example.tex
\section{テーマについて}
\subsection{ブロック}
\begin{frame}{\insertsubsection}
  なんかすごい手法\cite{Ono:2010:LVAICA,Ono:2011:WASPAA,Ono:2012:IWAENC,akhtar:2012:ISCAS,}
  \begin{columns}[T,onlytextwidth]
    \column{.5\linewidth}
      \begin{block}{block}
        block の色は\structure{\ruby[g]{群青}{ぐんじょう}色}
      \end{block}
      \begin{exampleblock}{exampleblock}
        example block の色は\textcolor{example text.fg}{なんかよくわからん緑色}
      \end{exampleblock}
      \begin{alertblock}{alertblock}
        alert block の色は\textcolor{alerted text.fg}{緋色}
      \end{alertblock}
    \column{.5\linewidth}
      \ambiset{block=fill}
      \begin{block}{block}
        block の色は\structure{\ruby[g]{群青}{ぐんじょう}色}
      \end{block}
      \begin{exampleblock}{exampleblock}
        example block の色は\textcolor{example text.fg}{なんかよくわからん緑色}
      \end{exampleblock}
      \begin{alertblock}{alertblock}
        alert block の色は\textcolor{alerted text.fg}{緋色}
      \end{alertblock}
      ブロックの背景を塗りつぶすこともできます
  \end{columns}
\end{frame}

{
  \setbeamertemplate{frametitle}[hidenumber]
  \begin{frame}[noframenumbering]{\insertsubsection}
    ページ番号を一時的に消すこともできます
    \uncover<2->{ページ数も矛盾しません!}
  \end{frame}
}

\subsection{Itemize}
\begin{frame}{\insertsubsection}
    見てくださいこのかわいいラベルたち
    \begin{itemize}
      \item[\pros] Pros
      \item[\cons] Cons
      \item[\todo] todo的なやつ
      \item[\done] todo完了的なやつ
      \item[\new] 新しいやつ
      \item[\goal] 「そこで,」的なやつ
    \end{itemize}
\end{frame}

\newcommand{\Obs}{\bm{x}}
\newcommand{\Src}{\bm{s}}
\newcommand{\Est}{\bm{y}}
\newcommand{\Mix}{\bm{A}}
\newcommand{\Sep}{\bm{W}}
\newcommand{\freq}{f}
\newcommand{\tframe}{t}
\newcommand{\ft}{\freq\tframe}
\begin{frame}{数式の装飾}
  \begin{equation*}
    \hlmath<2-5>{\Obs _{\ft}}  {観測信号}{(-1\zw,-1\zw)} {left} {structure.fg}{1.3em}%
    =
    \hlmath<4-5>{\Mix _{\freq}}{混合行列}{(1\zw,-2.2\zw)}{right}{alerted text.fg}{1.3em}%
    \hlmath<3-5>{\Src _{\ft}}  {音源信号}{(1\zw,-1\zw)}  {right}{example text.fg}{1.3em}%
  \end{equation*}

  \onslide<6->{
    \begin{equation*}
      \hlmath<6->{\Est _{\ft}}  {分離信号}{(-1\zw,-1\zw)} {left} {structure.fg}{1.3em}%
      =
      \hlmath<7->{\Sep _{\freq}}{分離行列}{(1\zw,-2.2\zw)}{right}{alerted text.fg}{1.3em}%
      \hlmath<8->{\Mix _{\ft}}  {観測信号}{(1\zw,-1\zw)}  {right}{example text.fg}{1.3em}%
    \end{equation*}
  }
\end{frame}
